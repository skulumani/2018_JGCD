\documentclass[journal]{new-aiaa}
%\documentclass[conf]{new-aiaa} for conference papers
\usepackage[utf8]{inputenc}

\usepackage{graphicx}
\usepackage{amsmath}
\usepackage[version=4]{mhchem}
\usepackage{siunitx}
\usepackage{longtable,tabularx}
\setlength\LTleft{0pt} 

\title{Real Time Adaptive Shape Reconstruction for Asteroid Landing}

\author{Shankar Kulumani \footnote{PhD, Mechanical and Aerospace Engineering, Address/Mail Stop, and AIAA Member Grade (if any) for first author.} and Taeyoung Lee \footnote{Associate Professor, Department Name, Address/Mail Stop, and AIAA Member Grade (if any) for second author.}}
\affil{Goerge Washington University, Washington, DC, 20052}

\begin{document}

\maketitle

\begin{abstract}
    Knowledge of the shape of an asteroid is crucial for spacecraft operations.
    The standard method of determining the gravitational potential, through the use of a polyhedron potential model, is dependent on the shape model.
    Furthermore, accurate landing or low altitude operations requires accurate knowledge of the surface topology. 
    The typical approach to shape determination requires an extensive ``mapping'' phase of the mission over which extensive measurements are collected and transmitted for Earth-based processing.
    Instead, we present an efficient method for estimating the shape of an asteroid in real time.
    Range measurements of the surface are used to incrementally correct an initial shape estimate according to Bayesian framework. 
    Then, an optimal guidance scheme is proposed to control the vantage point of the range sensor to construct a more accurate model of the asteroid shape. 
    This shape model is then used in a nonlinear controller to track a desired trajectory about the asteroid.
    % Finally, a multi resolution approach is presented which enables for a higher fidelity shape representation in a specified location while avoiding the inherent burdens of a uniformly high resolution mesh. 
    This approach enables for an accurate shape determination around a potential landing site.
    We demonstrate this approach using several radar shape models of asteroids and provide a full dynamic simulation about asteroid 4769 Castalia.
\end{abstract}

\section*{Nomenclature}

\noindent(Nomenclature entries should have the units identified)

{\renewcommand\arraystretch{1.0}
\noindent\begin{longtable*}{@{}l @{\quad=\quad} l@{}}
$A$  & amplitude of oscillation \\
$a$ &    cylinder diameter \\
$C_p$& pressure coefficient \\
$Cx$ & force coefficient in the \textit{x} direction \\
$Cy$ & force coefficient in the \textit{y} direction \\
c   & chord \\
d$t$ & time step \\
$Fx$ & $X$ component of the resultant pressure force acting on the vehicle \\
$Fy$ & $Y$ component of the resultant pressure force acting on the vehicle \\
$f, g$   & generic functions \\
$h$  & height \\
$i$  & time index during navigation \\
$j$  & waypoint index \\
$K$  & trailing-edge (TE) nondimensional angular deflection rate\\
$\Theta$ & boundary-layer momentum thickness\\
$\rho$ & density\\
\multicolumn{2}{@{}l}{Subscripts}\\
cg & center of gravity\\
$G$ & generator body\\
iso	& waypoint index
\end{longtable*}}




\section{Introduction}
% Motivation for missions/studying asteroids
Small solar system bodies, such as asteroids and comets, continue to remain a focus of scientific study.
The small size of these bodies prevents the formation of large internal pressures and temperatures which helps to preserve the early chemistry of the solar system.
This insight offers additional detail into the formation of the Earth and also of the probable formation of other extrasolar planetary bodies.
Of particular interest are those near-Earth asteroids (NEA) which inhabit heliocentric orbits in the vicinity of the Earth. 
These easily accessible bodies provide attractive targets to support space industrialization, mining operations, and scientific missions.
In spite of the significant interest, and the extensive research by the community, the operation of spacecraft near small bodies remains a challenging problem.

% dynamics are difficult around asteroids
The dynamic environment around asteroids is strongly perturbed and challenging for analysis and mission operations~\cite{scheeres2012}.
Due to their low mass, which in turn causes a low gravitational attraction, asteroids may have irregular shapes.
Furthermore, asteroids may also have a chaotic spin state due to the absorption and emittance of solar radiation~\cite{rubincam2000}.
As a result, approaches utilizing an inverse square gravitational model do not capture the  true dynamic environment.
In addition, the vast majority of asteroids are difficult to track and characterize using ground based sensors.
Due to their small size, frequently with a maximum radius less than \SI{1}{\kilo\meter}, and low albedo, the reflected energy of these asteroids is insufficient for reliable detection or tracking.
Therefore, the dynamics model of the asteroid is relatively coarse prior to in situ measurements from a dedicated spacecraft.
As a result, any spacecraft mission to an asteroid must include the ability to update the dynamic model given in situ measurements and remain robust to unmodelled forces.

Another key dynamic consideration is the coupling between rotational and translational states around the asteroid.
The coupling is induced due to the different gravitational forces experienced on various portions of the spacecraft. 
The effect of the gravitational coupling is related to the ratio of the spacecraft size and orbital radius~\cite{hughes2004}.
For operations around asteroids, the ratio is relatively large which causes a much larger coupling between the translational and rotational states.
References~\citenum{elmasri2005} and~\citenum{sanyal2004a} investigated the coupling of an elastic dumbbell spacecraft in orbit about a central body, but only considered the case of a spherically symmetric central body.
Furthermore, the spacecraft model is assumed to remain in a planar orbit.
As a result, these developments are not directly applicable to motion about an asteroid, which experiences highly non-Keplerian motion.
Reference~\citenum{misra2015b} investigated the effect of coupled motion on long term trajectories around asteroids.
However, the analysis only considered a second order spherical harmonic gravitational potential model. 
Therefore, these results are only valid when far from the asteroid surface and will diverge when used within the Brillouin sphere.

% Gravity model is important and dependent on shape
An accurate gravitational potential model is critical for performing low altitude and/or surface operations around asteroids.
Due to the irregular shape, trajectories will pass within the Brillouin sphere, where the typical spherical harmonic model diverges from the true gravitational potential.
The standard approach for asteroid missions is to compute the gravitational potential using a polyhedron potential model~\cite{werner1996}.
The polyhedron potential model provides the exact gravitational potential, and subsequently the gravitational acceleration, for a given triangular faceted shape model of an asteroid.
The method provides the exact potential at any point outside the body for a given shape model.
As a result, the accuracy of the gravitational potential is primarily dependent on the accuracy with which the shape model represents the true surface.
A high fidelity shape, which necessarily has many vertices and faces, is required for an accurate computation of the gravitational acceleration and enabling low altitude operations.

% Challenges involved in operation near asteroids (gravity, shape, distance)
% generating the shape from the ground is difficult
Prior to the arrival of a spacecraft at an asteroid, Earth-based sensors are used to characterize the body.
Using both optical and radar sensors allows for the precise orbit of the asteroid to be determined.
Another vital task is the determination of the asteroid shape from radar data~\cite{hudson1994,busch2011}.
This is a challenging problem as it requires the simultaneous estimation of the asteroid spin state and shape.
Furthermore, determining the shape from radar is currently the only Earth-based technique that can produce detailed three-dimensional shape information of near-Earth objects~\cite{greenberg2015}.
The current approach is based on an estimation scheme which iteratively perturbs a shape to match given radar data.
This computationally intensive approach is only able to capture the gross size and shape and is unable to capture the small surface features of the asteroid.
Frequently, only a coarse model is possible from the ground and an accurate shape must be determined only after a spacecraft has rendezvoused with the asteroid.
As a result, upon arrival the gravitational environment near the asteroid is poorly modeled as the shape of the asteroid is not accurate.
Therefore, the polyhedron potential model is not appropriate immediately upon arrival but rather only after the shape has been determined.

% on arrival spacecraft spend long periods mapping, and depending on mission this might be unallocable
On approach to an asteroid, spacecraft navigation and guidance is primarily based on ground measurements.
After arrival, a spacecraft will generally spend months or years in a mapping mission phase~\cite{kubota2003,cole1998}.
During this period, spacecraft sensors, such as on board optical telescopes or Light radio Detection and Ranging (LIDAR), are used to characterize the asteroid.
The resulting imagery and range data is transmitted to the ground and the resulting asteroid shape and motion is estimated. 
During this mapping phase the spacecraft must remain in a quiescent state devoted entirely to mapping the surface.
Depending on the mission type, this long period of mapping is crucial to the mission, such as sample collection~\cite{gates2015}. 
However, other missions, such as asteroid mitigation, may be severely limited by the time and ground resources required to generate a surface shape.
Furthermore, the long distances involved necessitate on-board autonomy to enable to spacecraft to operate without ground communications.
Similarly, during landing the spacecraft will require the ability to sense and model the surface topography in order to safely land in an unknown environment.
The dependency on expensive ground-based shape reconstruction techniques limits the ability of spacecraft to autonomously operate at asteroids.
Mitigation of this ground based surface modeling will greatly expand the range of missions possible.

% we want to generate the shape in real time and then use this shape for updated and better control
In this paper, we develop a method to compute the surface shape of an asteroid from range measurements.
Our approach is able to operate in real time and incrementally update the shape model of an asteroid as new range measurements are collected.
This approach allows for the shape to be continually updated as range measurements are used to locally modify the shape estimate.
Furthermore, this updated shape is then used in a nonlinear controller to enable the tracking of a landing trajectory to the surface.
In contrast to previous work, we explicitly consider the gravitational coupling between orbit and attitude dynamics.
Furthermore, instead of computationally expensive surface reconstruction methods, we present a straightforward and conceptually simple method to enable real time shape updates. 


% our approach for paper. Real time method to update asteroid shape model and use updated model in closed loop control

% Benefits and contribution of our approach
In short, this paper presents a method to incrementally update the shape  model of an asteroid from range measurements. 
Our approach alleviates the need for a dedicated mapping phase as the spacecraft is able to update its shape model in real time and without expensive computations.
This type of approach allows for the spacecraft to maneuver and land on the asteroid immediately upon arrival rather than spending several months mapping the surface.
This updated shape model is then used to in a nonlinear controller to track a desired state trajectory for the dynamics of a rigid body spacecraft.
The dynamics are developed on the nonlinear manifold of rigid body motions, namely the special euclidean group.
This formulation is based on an intrinsic geometric description of the motion and accurately captures the coupling between orbit and attitude dynamics. 
The presented approach allows for a spacecraft to transition directly from arrival to the surface while reconstructing the surface shape in real time.

\section{Problem Formulation}\label{sec:problem}

In this paper, we consider the motion of a dumbbell model of spacecraft around an asteroid.
The dumbbell model captures the important interactions of the coupling between orbital and attitude dynamics.
In this model, the spacecraft consists of two masses connected by a massless rod.
The asteroid is modeled as a constant density polyhedron with constant, and known, spin about its maximum moment of inertia. 
Without loss of generality, we define body fixed frames for both the spacecraft and asteroid, which are aligned with the principle axes of each body and originate at their respective center of mass. 
The kinematics of the dumbbell and asteroid are described in the inertial frame by
\begin{itemize}
    \item \( \vc{x} \) - the position of the center of mass of the dumbbell spacecraft represented in the inertial frame, \( \vc{e}_i\),
    \item \( R \) - the rotation matrix which transforms vectors defined in the spacecraft fixed frame, \( \vc{b}_i \), to the inertial frame, \( \vc{e}_i \),
    \item \( \vc{\Omega} \) - the angular velocity of the spacecraft body fixed frame relative to the inertial frame and represented in the dumbbell body fixed frame, \( \vc{b}_i \), and
    \item \( R_A \) - the rotation matrix which transforms vectors defined in the asteroid fixed frame, \( \vc{f}_i \), to the inertial frame, \( \vc{e}_i \).
\end{itemize}
In this work, we assume that the asteroid is much more massive than the spacecraft and its motion is not affected by that of the spacecraft.
This assumption allows us to treat the motion of the vehicle independently from that of the asteroid. 

\subsection{Spacecraft Dynamic Model}
Using Hamilton's principle one can derive the inertial equations of motion of the dumbbell spacecraft~\cite{kulumani2017b} as
\begin{align}
    \dot{\vc{x}} &= \vc{v}, \\
    \parenth{m_1 + m_2} \dot{\vc{v}} &= m_1 R_A \deriv{U}{\vc{z}_1} + m_2 R_A \deriv{U}{\vc{z}_2}, \\
    \dot{R} &= R S(\vc{\Omega}) , \\
    J \dot{\vc{\Omega}} + \vc{\Omega} \times J \vc{\Omega} &= \vc{M}_1 + \vc{M}_2.
\end{align}
The vectors \( \vc{z}_1 \) and \( \vc{z}_2\) define the position of the dumbbell masses in the asteroid fixed frame and are defined as
\begin{align}
    \vc{z}_1 &= R_A^T \parenth{\vc{x} + R \vc{\rho}_1} , \\
    \vc{z}_2 &= R_A^T \parenth{\vc{x} + R \vc{\rho}_2}.
\end{align}
The gravitational moment on the dumbbell \( \vc{M}_i\) is defined as
\begin{align}
    \vc{M}_i = m_i \parenth{S(R_A^T \vc{\rho}_i) R^T \deriv{U}{\vb{z}_i}}.
\end{align}
where the polyhedron potential is defined as 
\begin{align}
    U(\vc{r}) &= \frac{1}{2} G \sigma \sum_{e \in \text{edges}} \vc{r}_e \cdot \vc{E}_e \cdot \vc{r}_e \cdot L_e - \frac{1}{2}G \sigma \sum_{f \in \text{faces}} \vc{r}_f \cdot \vc{F}_f \cdot \vc{r}_f \cdot \omega_f,
\end{align}
and \( \vc{r}_e\) and \(\vc{r}_f \) are the vectors from the spacecraft to any point on the respective edge or face, \( G\) is the universal gravitational constant, and \( \sigma \) is the constant density of the asteroid.
The position of each mass \(m_i\) of the dumbbell is defined in the dumbbell fixed frame by the vector \(\vb{\rho}_i\). 


\subsection{Polyhedron Potential Model}\label{sec:polyhedron_potential}

% most use a spherical harmonic model or a ellipsoid model but we use a polyhedron model
An accurate gravitational potential model is necessary for the operation of spacecraft about asteroids.
Additionally, a detailed shape model of the asteroid is needed for trajectories passing close to the body.
The classic approach is to expand the gravitational potential into a harmonic series and compute the series coefficients.
However, the harmonic expansion is always an approximation as a result of the infinite order series used in the representation.
Additionally, the harmonic model used outside of the circumscribing sphere is not guaranteed to converge inside the sphere, which makes it unsuitable for trajectories near the surface.

We represent the gravitational potential of the asteroid using a polyhedron gravitation model.
This model is composed of a polyhedron, which is a three-dimensional solid body, that is defined by a series of vectors in the body-fixed frame.
The vectors define vertices in the body-fixed frame as well as planar faces which compose the surface of the asteroid.
We assume that each face is a triangle composed of three vertices and three edges.
As a result, only two faces meet at each edge while three faces meet at each vertex.
Only the body-fixed vectors, and their associated topology, is required to define the exterior gravitational model.
References~\cite{werner1994} and~\cite{werner1996} give a detailed derivation of the polyhedron model.
Here, we summarize the key developments and equations required for implementation.

Consider three vectors \( \vc{v}_1, \vc{v}_2, \vc{v}_3 \in \R^{3 \times 1} \), assumed to be ordered in a counterclockwise direction about an outward facing normal vector, which define a face.
For each face we define the face dyad \( \vc{F}_f \) as
\begin{align}\label{eq:face_dyad}
    \vc{F}_f &= \hat{\vc{n}}_f \hat{\vc{n}}_f \in \R^{3 \times 3}.
\end{align}
Each edge is a member of two faces and has an outward pointing edge normal vector, perpendicular to both the edge and the face normal.
For the edge connecting the vectors \( \vc{v}_1 \) and \( \vc{v}_2 \), which are shared between the faces \(A\) and \( B\), the per edge dyad is given by
\begin{align}\label{eq:edge_dyad}
    \vc{E}_{12} = \hat{\vc{n}}_A \hat{\vc{n}}_{12}^A + \hat{\vc{n}}_B \hat{\vc{n}}_{21}^B \in \R^{3 \times 3}.
\end{align}
The edge dyad \( \vc{E}_e  \), is defined for each edge and is a function of the two adjacent faces meeting at that edge.
The face dyad \( \vc{F}_f \), is defined for each face and is a function of the face normal vectors.

Let \( \vc{r}_i \in \R^{3 \times 1} \) be the vector from the spacecraft to the vertex \( \vc{v}_i \) and it's length is given by \( r_i = \norm{\vc{r}_i} \in \R^{1} \).
The per-edge factor \( L_e \in \R^{1}\), for the edge connecting vertices \( \vc{v}_i \) and \( \vc{v}_j \), with a constant length \( e_{ij} = \norm{\vc{e}_{ij}} \in \R^1\) is
\begin{align}\label{eq:edge_factor}
    L_e &= \ln \frac{r_i + r_j + e_{ij}}{r_i + r_j - e_{ij}}.
\end{align}
For the face defined by the vertices \( \vc{v}_i, \vc{v}_j, \vc{v}_k \) the per-face factor \( \omega_f \in \R^{1} \) is
\begin{align}\label{eq:face_factor}
    \omega_f &= 2 \arctan \frac{\vc{r}_i \cdot \vc{r}_j \times \vc{r}_k}{r_i r_j r_k + r_i \parenth{\vc{r}_j \cdot \vc{r}_k} + r_j \parenth{\vc{r}_k \cdot \vc{r}_i} + r_k \parenth{\vc{r}_i \cdot \vc{r}_j}} .
\end{align}
The gravitational potential due to a constant density polyhedron is given as
\begin{align}\label{eq:potential}
    U(\vc{r}) &= \frac{1}{2} G \sigma \sum_{e \in \text{edges}} \vc{r}_e \cdot \vc{E}_e \cdot \vc{r}_e \cdot L_e - \frac{1}{2}G \sigma \sum_{f \in \text{faces}} \vc{r}_f \cdot \vc{F}_f \cdot \vc{r}_f \cdot \omega_f \in \R^1,
\end{align}
where \( \vc{r}_e\) and \(\vc{r}_f \) are the vectors from the spacecraft to any point on the respective edge or face, \( G\) is the universal gravitational constant, and \( \sigma \) is the constant density of the asteroid.
Furthermore we can use these definitions to define the attraction, gravity gradient matrix, and Laplacian as
\begin{align}
    \nabla U ( \vc{r} ) &= -G \sigma \sum_{e \in \text{edges}} \vc{E}_e \cdot \vc{r}_e \cdot L_e + G \sigma \sum_{f \in \text{faces}} \vc{F}_f \cdot \vc{r}_f \cdot \omega_f \in \R^{3 \times 1} , \label{eq:attraction}\\
    \nabla \nabla U ( \vc{r} ) &= G \sigma \sum_{e \in \text{edges}} \vc{E}_e  \cdot L_e - G \sigma \sum_{f \in \text{faces}} \vc{F}_f \cdot \omega_f \in \R^{3 \times 3}, \label{eq:gradient_matrix}\\
    \nabla^2 U &= -G \sigma \sum_{f \in \text{faces}}  \omega_f \in \R^1 .\label{eq:laplacian}
\end{align}

One interesting thing to note is that both~\cref{eq:face_dyad,eq:edge_dyad} can be precomputed without knowledge of the position of the satellite.
They are both solely functions of the vertices and edges of the polyhedral shape model and are computed once and stored.
Once a position vector \( \vc{r} \) is defined, the scalars given in~\cref{eq:edge_factor,eq:face_factor} can be computed for each face and edge.
Finally,~\cref{eq:potential} is used to compute the gravitational potential on the spacecraft.
The Laplacian, defined in~\cref{eq:laplacian}, gives a simple method to determine if the spacecraft has collided with the body~\cite{werner1996}. 

\section{Incremental Shape Reconstruction}\label{sec:radius_update}

One of the first tasks for any spacecraft mission to a small body is to generate an estimate of the shape.
We assume that upon arrival at a target body, the spacecraft contains an initial estimate for the shape of the small body.
This shape can be a coarse estimate computed from ground measurements or it can be a triaxial ellipsoid based on the semimajor axes of the asteroid.
Additionally, we assume that the shape estimate is closed and a triangular faceted surface mesh, emulating those used in practice to represent asteroids.
Furthermore, the number of vertices in the estimate can be scaled according to the desired final accuracy or computational capabilities.

We assume the spacecraft contains a range sensor, such as LIDAR, that allows for the accurate measurement of the relative distance between the spacecraft and asteroid~\cite{zuber1997,zuber2000}.
This type of sensor measures the round-trip time for a pulse of energy to leave the spacecraft, reflect off the surface, and return to a collector on board.
Given the time total time of flight, the distance can be accurately computed using \( d = \frac{\Delta TOF}{2 c} \) where \( c = \SI{2.998e8}{\meter\per\second}\) is the constant speed of light.
Assuming accurate knowledge of the pointing direction of the spacecraft, in the form of the rotation matrix \( R \in \R^{3 \times 3 } \), we can compute a direction from the spacecraft to the measurement location on the surface.
The output of this sensor is a vector, \( \vc{d}_i \), defined in the spacecraft fixed frame which gives the direction to a measurement point on the surface. 
Using the state of the asteroid, we can transform this measurement to the asteroid fixed frame using the simple transformation
\begin{align*}
    \vc{p}_i = R_A^T R \vc{d}_i .
\end{align*}
Given many measurements, \( \vc{p}_i \in \R^3 \), of the asteroid surface we can efficiently update our initial shape estimate to that of the true surface.
\Cref{fig:lidar_example} shows asteroid 4769 Castalia and a representation of several LIDAR measurements. 
The spacecraft measures the range between itself and the asteroid surface to several points within the field of view of the sensor. 
These measurements provide a collection of points which lie on the surface of the asteroid, and by combining many points, a so called ``point cloud'', allows us to reconstruct the shape.
\begin{figure}
    \centering
    \includegraphics[width=0.75\textwidth]{figures/castalia_raycasting_plot.jpg}
    \caption{Simulated LIDAR measurements of asteroid Castalia~\label{fig:lidar_example}}
\end{figure}

\paragraph{Bayesian Shape Update}

Our algorithm applies a Bayesian framework to radially modify each vertex \( \vc{v}_i \in \R^3\) of the shape estimate based on measurement \( \vc{p}_i \in \R^3 \). 
This approach alleviates much of complexity of incorporating new vertices or surface triangulation common in surface reconstruction methods~\cite{berg2008}.
This assumption means that the total number of vertices of the shape model is fixed.
However, additional detail, in the form of additional vertices, is possible by using standard mesh subdivision algorithms~\cite{orourke1998}.

The radial distance of each vertex, \( v_i = \norm{\vc{v}_i}\), is assumed to be distributed according to the Gaussian distribution
\begin{align*}
    v_i \sim \mathcal{N}(r_i, w_i^2)
\end{align*}
where \( r_i \) is the initial estimate of the radial distance of vertex \( \vc{v}_i\) and \( w_i \) is the initial variance, or confidence, in the radial distance.
The radial distance of each measurement, \( p_{j,i} = \norm{\vc{p}_j}\), is also assumed to be distributed according to the Gaussian distribution
\begin{align*}
    p_{j,i} \sim \mathcal{N}(r_{j,i}, w_{j,i}^2)
\end{align*}
where \( r_{j,i} = \norm{\vc{p}_{j,i}} \) defines the radial distance of the surface vector measurement and \( w_{j, i}\) defines the variance of the measurement with respect to vertex \( \vc{v}_i\).
Each measurement is defined by the index \( j \) while the associated vertex is defined by \( i \). 
As a result, the measurement \( p_{j, i} \) defines the distribution of measurement \( j \) with respect to vertex \( i \). 
Any given measurement may be used to update one or several vertices.

The variance for each measurement vector is assumed to be related to the ``distance'' from the measurement to vertex \( \vc{v}_i \).
Here, we use the geodesic distance to parameterize the difference, and hence  uncertainty, of associating the measurement with a given vertex.
From spherical trigonometry~\cite{gade2010}, the central angle between measurement \( \vc{p}_i \) and vertex \( \vc{v}_i \) of the shape estimate is
\begin{align}\label{eq:geodesic_distance}
    \Delta \sigma_{j,i} = \arctan \parenth{\frac{\norm{\vc{p}_j \times \vc{v}_i}}{\vc{p}_j \cdot \vc{v}_i }}.
\end{align}
The variance of measurement \( \vc{p}_i \) with respect to vertex \( \vc{v}_i \) is then defined by the geodesic distance as
\begin{align}
    w_{j, i} = \norm{\vc{p}_j} \Delta \sigma_{j,i} .
\end{align}
This approach relates the uncertainty of the measurement \( \vc{p}_j \) with the geodesic distance to a given vertex, \( \vc{v}_i \).
As a result, measurements which are far from a vertex, i.e.\ \( \Delta \sigma \) is large, will tend to have a larger variance and hence uncertainty. 
This approach can be considered as a form of a correlation based sensor model~\cite{thrun2005}.
The main benefit of a correlation based approach, in contrast to feature extraction is the relative simplicity of implementation.
However, the resulting correlation values do not posses any physical significance and do not represent the noise or uncertainty characteristics of the sensor.

From Bayes' theorem, the posterior probability is defined as
\begin{align}
    p(v_i | p_{j, i}) = \frac{p(p_{j, i} | v_i) p(v_i)}{p( p_{j, i})} \propto p(p_{j,i} | v_i) p(v_i).
\end{align}
From the properties of a Gaussian, the posterior probability given a measurement is also distributed according to a Gaussian distribution~\cite{bishop2006} and given by
\begin{align}\label{eq:posterior_probability}
    \mathcal{N} \parenth{\frac{w_{j, i}^2 r_i + w_i^2 r_{j, i}}{w_i^2 + w_{j, i}^2} , \frac{w_i^2  w_{j, i}^2}{w_i^2 +  w_{j, i}^2}} .
\end{align}
From~\cref{eq:posterior_probability}, the posterior probability conditioned on the measurement is the weighted average of the prior knowledge and the measurement. 
Measurements that are far from the vertex will have a high uncertainty or variance and will have a reduced impact on the radial position of the vertex.
Additional measurements are incorporated using a weighted average of prior belief and the measurement uncertainty.

In order to improve the computational efficiency measurement updates are assumed to be local in nature.
Instead of applying a measurement to all vertices of the mesh, the measurement is only applied to the vertices which are within a specified region of the measurement. 
We define a region of interest, \( \Delta S \), about each measurement which defines the surface area over which the measurement will affect the mesh estimate.
We relate \( \Delta S \) to an equivalent angular constraint using
\begin{align}\label{eq:region_of_interest}
    \Delta \sigma_{max} = \sqrt \frac{\Delta S}{r_b^2}
\end{align}
% shankar: brillouin is correct. https://arc.aiaa.org/doi/10.2514/6.2014-4302
where \( r_b \) defines the Brillouin  sphere radius, or the radius of the circumscribing sphere of the asteroid.
Only vertices which satisfy \( \Delta \sigma_i \leq \Delta \sigma_{max} \) are considered in the Bayesian update shown in~\cref{eq:posterior_probability}.

The approach presented in this section allows one to update the shape of small body given a single range measurement of the surface.
A sequential process can be used to iteratively update the shape estimate given many measurements of the surface. 

\section{Optimal Guidance for Shape Reconstruction}\label{sec:explore_asteroid}

The mesh update algorithm presented does not offer a method to determine which portion of the surface needs to be updated. 
In this section, we present an optimal approach for the guidance of the vehicle in order to update the shape estimate.
A nonlinear geometric controller is utilized which allows the spacecraft to maneuver to the best location that will update the shape estimate~\cite{kulumani2017b}.
This guidance schemes enables autonomous operations around a small body.

We define a cost associated with each vertex \( \vc{v}_i \) of the shape estimate
\begin{align}\label{eq:explore_cost}
    J_i (x, R, R_A) = \alpha_w J_{w_i} + \alpha_d J_{d_i}(x_r) + \alpha_c J_{c_i}(x_r)
\end{align}
where the weighting factors \( \alpha_w, \alpha_d, \alpha_c \in \R^1 \) are chosen such that \( \alpha_w + \alpha_d + \alpha_c = 1 \).
The cost function is defined as a function of the current inertial position, \( x \in \R^3 \), and the attitude, \( R \in \R^{3\times3}\) of the spacecraft.
Furthermore, knowledge of the small body rotation is required in order to determine the position of the spacecraft in the small body fixed frame, \( x_r = R_A^T x\).

The term \( J_{w_i} \in \R^1 \) represents the cost associated with the uncertainty of vertex \( i \) as
\begin{align}\label{eq:weight_cost}
    J_{w_i} &= - \frac{w_i}{w_m}
\end{align}
where \( w_i \) is the uncertainty of vertex \( i \) and \( w_m \) is a maximum uncertainty used to scale the values.
The term \( J_{d_i} \) represents the scaled geodesic distance between the current state of the spacecraft and vertex \( i \),
\begin{align}\label{eq:distance_cost}
    J_{d_i}(\rpos) &= \frac{1}{\pi} \arctan \parenth{ \frac{\norm{\rpos \times \vc{v}_i}}{\rpos \cdot \vc{v}_i}}.
\end{align}

Finally, a control component is included in the cost function which penalizes vertices that are difficult to reach.
Consider, the current position of the spacecraft in the small body fixed frame as \( \rpos\) and a desired vertex \( \vc{v}_i \) of the shape estimate.
We can define a normal vector to the plane spanned by \( \rpos, \vc{v}_i \) as
\begin{align}\label{eq:normal_to_plane}
    \vc{n}_i = \frac{\rpos \times \vc{v}_i}{\norm{\rpos} \norm{\vc{v}_i}}.
\end{align}
Then a trajectory \( x_d(\theta) \) as
\begin{align}\label{eq:spherical_waypoint}
    x_d(\theta) = r_d \exp{\parenth{\theta \hat{\vc{n}_i}} } \frac{\rpos}{\norm{\rpos}},
\end{align}
where \( \theta : \bracket{0, \frac{\rpos \cdot \vc{v}_i}{\norm{\rpos}\norm{\vc{v}_i}}} \to \R^1\) parameterizes the desired trajectory.
\Cref{eq:spherical_waypoint} simply describes a portion of a great circle trajectory between the current state, \( \rpos \), and the desired vertex \( \vc{v}_i \)~\cite{chen2016}.
The altitude of the spacecraft, \( r_d \in \R \), can be chosen based on sensor characteristics of safety concerns.
For example, \( r_d \) can be chosen as the distance of the Biroullin sphere with an additional safety margin to mitigate any surface collision~\cite{scheeres2012a}.
We assume that the tracking errors are small, such that \( e_x, e_v \) are negligible, therefore the control becomes
\begin{align}\label{eq:tracking_control_cost}
    u_f(\theta) = -F_{ext}(x_d(\theta)), 
\end{align}
where the external force is defined by the polyhedron potential model given in~\cref{eq:attraction}.
The control cost is then defined as the integral over the desired trajectory~\cref{eq:spherical_waypoint} between the current state and the desired vertex as
\begin{align}\label{eq:control_cost}
    J_{c_i}(\rpos) = \frac{1}{u_m} \int_{\theta_0}^{\theta_f} u_f(\theta)^T R u_f(\theta) d\theta,
\end{align}
where \( u_m \) is used to normalize and scale \( J_{c_i} \).
\Cref{eq:control_cost} is numerically integrated over the trajectory \( x_d(t) \) and used to penalize vertices which have a larger cost.

The vertex which minimizes~\cref{eq:explore_cost} 
\begin{align*}
    \vc{v}_{min} = \min_{i} J_i(x, R, R_A),
\end{align*}
is determined and used to determine the optimum vertex of the shape model to measure.
This vertex is then used to determine the required state of the spacecraft in order to collect a measurement.
We assume that the spacecraft will maneuver to a location directly above the selected vertex, \( \vc{v}_{min} \), and point in the nadir direction to collect a measurement.
The desired state is transformed into the inertial frame and chosen as a location above \( \vc{v}_{min} \) as
\begin{align}
    x_d = r_d R_A \vc{v}_{min}, 
\end{align}
where \( r_d \) is again chosen to ensure a safety margin above the surface.
The desired attitude command, \( R_d\), is chosen such that the spacecraft camera axis, \( \vc{b}_1 \), is directed along the nadir towards the small body.
It is sufficient to define two orthogonal vectors to uniquely determine the attitude of the spacecraft.
The \( \vc{b}_{3d} \) vector is chosen to lie in the plane spanned by \(\vc{b}_{1d} \) and \( \vc{e}_3 = \vc{f}_3 \).
The desired attitude command is defined as
\begin{align}
    \vc{b}_{1d} &= - \frac{\vc{x}}{\norm{\vc{x}}} , \\
    \vc{b}_{3d} &= \frac{\vc{f}_3 - \parenth{\vc{f}_3 \cdot \vc{b}_{1d}} \vc{b}_{1d}}{\norm{\vc{f}_3 - \parenth{\vc{f}_3 \cdot \vc{b}_{1d}} \vc{b}_{1d}}}, \\
    \vc{b}_{2d} &= \vc{b}_{3d} \times \vc{b}_{1d} , \\
    R_d &= \begin{bmatrix} \vc{b}_{1d} & \vc{b}_{2d} & \vc{b}_{3d} \end{bmatrix} .
\end{align}
This form of \( R_d \) will direct the \( b_1 \) axis towards the small body, and can be modified for a different camera orientation.




\subsection{Headings}
Format the title of your paper in bold, 18-point type, with capital and lower-case letters, and center it at the top of the page. The names of the authors, business or academic affiliation, city, and state/province follow on separate lines below the title. The names of authors with the same affiliation can be listed on the same line above their collective affiliation information. Author names are centered, and affiliations are centered and in italic type. The affiliation line for each author includes that author’s city, state, and zip/postal code (or city, province, zip/postal code and country, as appropriate). The first footnote (bottom of first page) contains the job title and department name, and AIAA member grade for each author. Author email addresses may be included also.

Major headings in the template (``sections'' in the \LaTeX{} template commands) are bold 11-point font and centered. Please omit section numbers before all headings unless you refer frequently to different sections. Use Roman numerals for major headings if they must be numbered.

Subheadings (``subsections'' in the \LaTeX{} template commands) are bold, flush left, and either unnumbered or identified with capital letters if necessary for cross-referencing sections within the paper. There must be at least 2 of all subheadings and sub-subheadings. If there is only a single subheading or sub-subheading, please italicize the title of the subheadings, followed by a period, and run it into the text paragraph. 

Sub-subheadings (``subsubsections'' in the \LaTeX{} template commands) are italic, flush left, and either unnumbered or numbered with Arabic numerals (1, 2, 3, etc.) if necessary for cross-referencing sections within the paper.


\subsection{Abstract}
An abstract appears at the beginning of Full-Length Papers, Regular Articles, and Express Articles. (Survey and Design Forum Papers, History of Key Technologies Papers, invited lectures, and Technical/Engineering Notes do not include abstracts.) The abstract is one paragraph (not an introduction) and complete in itself (no reference numbers). It should indicate subjects dealt with in the paper and state the objectives of the investigation. Newly observed facts and conclusions of the experiment or argument discussed in the paper must be stated in summary form; readers should not have to read the paper to understand the abstract. Format the abstract bold, indented 3 picas (1/2 in.) on each side, and separated from the rest of the document by two blank lines.


\subsection{Nomenclature}
Papers with many symbols may benefit from a nomenclature list that defines all symbols with units, inserted between the abstract and the introduction. If one is used, it must contain all the symbology used in the manuscript, and the definitions should not be repeated in the text. In all cases, identify the symbols used if they are not widely recognized in the profession. Define acronyms in the text, not in the nomenclature. 

\subsection{Biographies}
Survey Papers and some Full-Length Papers include author biographies. These biographies are one paragraph each and should use the abstract formatting style.

\subsection{Footnotes and References}
Footnotes, where they appear, should be placed above the 1'' margin at the bottom of the page. To insert footnotes into the template, use the Insert>Footnote feature from the main menu as necessary. Footnotes are formatted automatically in the template, but if another medium is used, should appear in superscript as symbols in the sequence, *, $\dag$, $\ddag$, \S, \P, **, $\dag\dag$, $\ddag\ddag$, \S\S, etc.

List and number all references at the end of the paper. Corresponding bracketed numbers are used to cite references in the text \cite{vatistas1986reverse}, including citations that are an integral part of the sentence (e.g., ``It is shown in \cite{dornheim1996planetary} that\ldots '') or follow a mathematical expression: ``$A^{2} + B = C$ (Ref.~\cite{terster1997nasa}).'' For multiple citations, separate reference numbers with commas \cite{peyret2012computational,oates1997aerothermodynamics}, or use a dash to show a range \cite{volpe1994techniques,thompsonspacecraft,chi1993fluid}. Reference citations in the text should be in numerical order.

In the reference list, give all authors' names; do not use ``et al.'' unless there are six authors or more. Papers that have not been published should be cited as ``unpublished''; papers that have been submitted or accepted for publication should be cited as ``submitted for publication.'' Private communications and personal website should appear as footnotes rather than in the reference list.

References should be cited according to the standard publication reference style (for examples, see the ``References'' section of this template). Never edit titles in references to conform to AIAA style of spellings, abbreviations, etc. Names and locations of publishers should be listed; month and year should be included for reports and papers. For papers published in translation journals, please give the English citation first, followed by the original foreign language citation.

\subsection{Figures and Tables}
Insert tables and figures within your document; they may be either scattered throughout the text or grouped all together at the end of the file. Do not insert your figures in text boxes. Figures should have no background, borders, or outlines. In the \LaTeX{} template, use the ``caption'' command to type caption text. Captions are bold with a single tab (no hyphen or other character) between the figure number and figure description. See the Table 1 example for table style and column alignment. If you wish to center tables that do not fill the width of the page, simply highlight and “grab” the entire table to move it into proper position.


\begin{table}[hbt!]
\caption{\label{tab:table1} Transitions selected for thermometry}
\centering
\begin{tabular}{lcccccc}
\hline
& Transition& & \multicolumn{2}{c}{}\\\cline{2-2}
Line& $\nu''$& & $J'' $& Frequency, cm$^{-1}$& $FJ$, cm$^{-1}$& $G\nu $, cm$^{-1}$\\\hline
a& 0& P$_{12}$& 2.5& 44069.416& 73.58& 948.66\\
b& 1& R$_{2}$& 2.5& 42229.348& 73.41& 2824.76\\
c& 2& R$_{21}$& 805& 40562.179& 71.37& 4672.68\\
d& 0& R$_{2}$& 23.5& 42516.527& 1045.85& 948.76\\
\hline
\end{tabular}
\end{table}


Line drawings must be clear and sharp. Make sure that all lines and graph points are dark and distinct and that lettering is legible; 8- to 10-point type is suitable for artwork that is sized to fit the column width (3 ¼ in.). Keep the lettering size and style uniform both within each figure and throughout all of your illustrations. Place figure captions below each figure, and limit caption length to 20-25 words. If your figure has multiple parts, include the labels “a),” “b),” etc., below and to the left of each part, above the figure caption. Please verify that the figures and tables you mention in the text actually exist. When citing a figure in the text, use the abbreviation “Fig.” except at the beginning of a sentence. Do not abbreviate “Table.” Number each different type of illustration (i.e., figures and tables) sequentially with relation to other illustrations of the same type.

Figure labels must be legible after reduction to column width (preferably 8--10 points after reduction).

All tables are numbered consecutively and must be cited in the text; give each table a definitive title. Be sure that you have a minimum of two columns (with headings) and two rows to constitute a proper table; otherwise reformat as a displayed list or incorporate the data into the text. Plan tables to fit the column width (3 ¼ in.) or the journal page width (7 in.). Position a double rule at the top and bottom of each table and single rule under the column headings; do not use shading, border lines, or vertical rules between table columns. Position each table in the text close to where it is cited


\subsection{Equations}
Equations are numbered consecutively, with equation numbers in parentheses flush right, as in Eq.~\eqref{sample:equation}. Insert a blank line on either side of the equation. To insert an equation into the \LaTeX{} document, use the \verb|\begin{equation}...\end{equation}| command environment.

A sample equation is included here, formatted using the preceding instructions:

\begin{equation}
\label{sample:equation}
\int^{r_2}_0 F(r,\varphi){\rm d}r\,{\rm d}\varphi = [\sigma r_2/(2\mu_0)]\int^{\infty}_0\exp(-\lambda|z_j-z_i|)\lambda^{-1}J_1 (\lambda r_2)J_0 (\lambda r_i\,\lambda {\rm d}\lambda)
\end{equation}

Be sure that symbols in your equation are defined in the Nomenclature or immediately following the equation. Also define abbreviations and acronyms the first time they are used in the main text. (Very common abbreviations such as AIAA and NASA, do not have to be defined.)

\subsection{General Grammar and Preferred Usage}
Use only one space after periods or colons. Hyphenate complex modifiers: ``zero-field-cooled magnetization.'' Insert a zero before decimal points: ``0.25,'' not ``.25.'' Use ``\si{\centi\meter\squared}'' not ``cc.'' 

A parenthetical statement at the end of a sentence is punctuated outside of the closing parenthesis (like this). (A parenthetical sentence is punctuated within parenthesis.) Use American, not English, spellings (e.g., “color,” not “colour”). The serial comma is preferred: “A, B, and C” instead of “A, B and C.”

Be aware of the different meanings of the homophones “affect” (usually a verb) and “effect” (usually a noun), “complement” and “compliment,” “discreet” and “discrete,” “principal” (e.g., “principal investigator”) and “principle” (e.g., “principle of measurement”). Do not confuse “imply” and “infer.”

\section{Conclusion}
Although a conclusion may review the main points of the paper, it must not replicate the abstract. A conclusion might elaborate on the importance of the work or suggest applications and extensions. Do not cite references in the conclusion. Note that the conclusion section is the last section of the paper to be numbered. The appendix (if present), funding information, other acknowledgments, and references are listed without numbers.

\section*{Appendix}

An Appendix, if needed, appears \textbf{before} research funding information and other acknowledgments.

\section*{Funding Sources}

Sponsorship information and acknowledgments of financial support should be included here. \textbf{Authors are responsible for accurately reporting funding data relevant to their research.} Please confirm that you have correctly entered \textbf{all sources} of funding and grant/award numbers \textbf{for all authors} in this section of your article. You will also be asked to select the appropriate funding organization from a drop-down menu in ScholarOne when you submit your manuscript. Be careful to choose the correct funder name, as organization names can be similar, and also be mindful to select sub-organizations within the registry hierarchy that are the actual funding sources, as appropriate, rather than choosing the name of the parent organization. Information provided in your manuscript must match the funding data entered in ScholarOne.

\section*{Acknowledgments}
An Acknowledgments section, if used, \textbf{immediately precedes} the References. Individuals other than the authors who contributed to the underlying research may be acknowledged in this section. The use of special facilities and other resources also may be acknowledged. 

\bibliography{sample}

\end{document}
