% use paper, or submit
% use 11 pt (preferred), 12 pt, or 10 pt only

%\documentclass[letterpaper, preprint, paper,11pt]{AAS}	% for preprint proceedings
\documentclass[letterpaper, paper,11pt]{AAS}		% for final proceedings (20-page limit)
%\documentclass[letterpaper, paper,12pt]{AAS}		% for final proceedings (20-page limit)
%\documentclass[letterpaper, paper,10pt]{AAS}		% for final proceedings (20-page limit)
%\documentclass[letterpaper, submit]{AAS}			% to submit to JAS

\usepackage{AAS_packages}
%\usepackage{subfigure} % have subcaption in use instead
%\usepackage[notref,notcite]{showkeys}  % use this to temporarily show labels

\PaperNumber{XX-XXX}

\begin{document}

\title{Real Time Shape Reconstruction for Near Earth Asteroid Landing}

\author{Shankar Kulumani and Taeyoung Lee\thanks{Mechanical and Aerospace Engineering, George Washington University, 800 22nd St NW, Washington, DC 20052, Tel: 202-994-8710, Email: \href{mailto:skulumani@gwu.edu}{\{skulumani,tylee\}@gwu.edu}.}
}


\maketitle{} 		

\begin{abstract}
    Abstract for paper
\end{abstract}

\section{Introduction}\label{sec:introduction}
% Motivation for missions/studying asteroids
Small solar system bodies, such as asteroids and comets, continue to remain a focus of scientific study.
The small size of these bodies prevents the formation of large internal pressures and temperatures which help to preserve the early chemistry of the solar system.
This insight offers additional detail into the formation of the Earth and also of the probable formation of other extrasolar planetary bodies.
Of particular interest are those near-Earth asteroids (NEA) which inhabit heliocentric orbits in the vicinity of the Earth. 
These easily accessible bodies provide attractive targets to support space industrialization, mining operations, and scientific missions.
In spite of the significant interest, and the extensive research by the community, the operation of spacecraft near small bodies remains a challenging problem.

% dynamics are difficult around asteroids
The dynamic environment around asteroids is strongly perturbed and challenging for analysis and mission operations~\cite{scheeres2012}.
Due to their low mass, which in turn causes a low gravitational attraction, asteroids may have irregular shapes.
Furthermore, asteroids may also have a chaotic spin state due to the absorption and emmitance of solar radiation~\cite{rubincam2000}.
As a result, approaches utilizing an inverse square gravitational model do not capture the  true dynamic environment.
In addition, the vast majority of asteroids are difficult to track and characterize using ground based sensors.
Due to their small size, frequently less than \SI{1}{\kilo\meter}, and low albedo, the reflected energy of these asteroids is insufficient for reliable detection or tracking.
Therefore, the dynamics model of the asteroid is relatively coarse prior to insitu measurements from a dedicated spacecraft.
As a result, any spacecraft mission to an asteroid must include the ability to update the dynamic model given insitu measurements and robustness to unmodelled forces.

Another key dynamic consideration is the coupling between rotational and translational states around the asteroid.
The coupling is induced due to the different gravitational forces experienced on various portions of the spacecraft. 
The effect of the gravitational coupling is related to the ratio of the spacecraft size and orbital radius~\cite{hughes2004}.
For operations around asteroids, the ratio is relatively large which causes a much larger coupling between the translational and rotational states.
References~\cite{elmasri2005} and~\cite{sanyal2004a} investigated the copuling of an elastic dumbbell spacecraft in orbit about a central body, but only considered teh case of a spherically symmetric central body.
Furthermore, the spacecraft model is assumed to remain in a planar orbit.
As a result, these developments are not directly applicable to motion about an asteroid, which experiences highly non-Keplerian motion.
Reference~\cite{misra2015b} investigated the effect of coupled motion on long term trajectories around asteroids.
However, the analysis only considered a second order spherical harmonic gravitational potential model. 
Therefore, these results are only valid when far from the asteroid surface and will diverge when used within the Brillouin sphere.

% Gravity model is important and dependent on shape
An accurate gravitational potential model is critical for performing low alititude and/or surface operations around asteroids.
Due to the irregular shape, trajectories will pass within the Brillouin sphere where the typical spherical harmonic model diverges from the true gravitational potential.
With respect to asteroid missions, the standard approach to compute the gravitational potential is the polyhedron potential model~\cite{werner1996}.
The polyhedral potential model provides the exact gravitational potential, and subsequently the gravitational acceleration, for a given triangular faceted shape model of an asteroid.
The method provides the exact potential for a given faceted shape model.
As a result, the accuracy of the gravitational potential is a function of the accuracy with which the shape modelrepresents the true surface topology.
A high fidelity shape, which necessarily has many vertices and faces, is required for an accurate computation of the true gravitational acceleration.
The determination of the shape of the asteroid surface is crucial for both determining the gravitational potential as well as operations near the surface.

% generating the shape from the ground is difficult

% on arrival spacecraft spend long periods mapping, and depending on misison this might be unallowable

% we want to generate the shape in real time and then use this shape for updated and better control

% Challenges invovled in operation near asteroids (gravity, shpae, distance)


% Current approach relies on long periods of mapping to generate the shape

% our approach for paper. Real time method to update asteroid shape model and use updated model in closed loop control

% Benefits and contribution of our approach


\begin{enumerate}
    \item Spacecraft missions around asteroid require long periods of mapping
        \begin{itemize}
            \item Several months/years are spent in a quiescent state devoted solely to mapping the surface
            \item Depending on the mission this time spent mapping is unnecessary and resource limiting - such as impact mitigation misions
            \item The main goal of this mapping process is to generate an accurate shape model of the asteroid and measure the surface features.
            \item The current methods rely on sending all data to the ground in order to compute the shape. 
            \item Furthermore this limits the abiilty  of teh spacecraft to operate without  reliable Earth communications and support
        \end{itemize}
    \item The surface shape is necessary in order to allow the spacecraft to operate near the surface of the a steroid
        \begin{itemize}
            \item The polyhedron potential model is a function of the shape and highly dependent on an accurate shape model.
            \item It is not possible to accurately determine the shape of the asteroid from ground based measurements.
                Due to the limitations of optical and radar measurements the typical best shape approximation prior to arrival is a coarse ellispoid shape CITE a paper.
            \item An accurate shape model is critical for not only the gravitational model but also to enable accurate landing operations.
                An accurate model of the surface is required for the gravity model, surface landing operations, and obstacle avoidance.
        \end{itemize}
    \item Due to the previous issues it is critical to be able to generate the shape of the asteroid surface accurately and quickly.
        \begin{itemize}
            \item This work develops a method to efficiently determine the shape of the an asteroid given LIDAR range measurements of the surface.
            \item The shape is updated in near real time (QUANTIFY THIS IN SOME MEANINGFUL FASHION) to  enable:
                \begin{itemize}
                    \item Updated gravity/dynamic model 
                    \item Utilize this updated dynamic model in the spacecraft controller
                    \item Allow for simulataenous shape reconstruction and spacecraft control
                \end{itemize}
            \item This alleviates teh need for a long term mapping campaign
        \end{itemize}
\end{enumerate}

\section{Asteroid shape representation}


\section{Problem Formulation}\label{sec:problem}
In this paper, we consider the landing of a dumbbell model of an under actuated spacecraft onto an asteroid.
The dumbbell model is a well-known representation of a multi body spacecraft.
Furthermore, the dumbbell model captures the important interactions of the coupling between orbital and attitude dynamics. 
The dumbbell spacecraft consists of two masses connected by a massless rod.
This simple model is useful to capture the main characteristics of a wide variety of spacecraft configurations.
Typically, spacecraft have mass concentrated in a central structure, referred to as the bus, which houses the command and control system, actuators, fuel, sensors etc. 
In addition, comparatively light-weight solar panels extend from the bus to provide electrical energy from solar radiation. 
As a result, the distributed mass of the spacecraft is captured with the dumbbell representation.

% most use a spherical harmonic model or a ellipsoid model but we use a polyhedron model
An accurate gravitational potential model is necessary for the operation of spacecraft about asteroids.
Additionally, a detailed shape model of the asteroid is needed for trajectories passing close to the body.
The classic approach is to expand the gravitational potential into a harmonic series and compute the series coefficients.
However, the harmonic expansion is always an approximation as a result of the infinite order series used in the representation.
Additionally, the harmonic model used outside of the circumscribing sphere is not guaranteed to converge inside the sphere, which makes it unsuitable for trajectories near the surface.
Instead, we use the polyhedron potential model to represent the shape of the asteroid and also the gravitational potential assuming a constant density~\cite{werner1994,werner1997}. 

The equations of motion of the dumbbell spacecraft about an asteroid are derived using Hamilton's Principle. 
The first step is to choose generalized coordinates \( q \) and a corresponding configuration space \( Q \).
The Lagrangian is then formed as the difference between the kinetic and potential energy in terms of our generalized coordinates. 
Hamilton's principle then states that the variation of the action integral
\begin{align}
    \mathsf{G} = \int_{t_0}^{t_f} T(\dot{q}) - V(q) dt,
\end{align}
is stationary with fixed endpoints. 
From this we obtain the Euler-Lagrange equations for the system.
Applying the Legendre transformation allows for the same dynamics to be expressed in an equivalent form as Hamilton's equations~\cite{lanczos1970}.

There is a key issue that arises in applying this process for the coupled motion about an asteroid.
The configuration space for rigid body motion is the semi-direct product, \(\SE = \R^3 \rtimes \SO \), namely the special euclidean group.
The variations should be carefully constructed such that they respect the geometry of the configuration space.
By expressing the motion of the dumbbell directly on the special euclidean group, we avoid the issues inherent in using other kinematic representations which fail to preserve the geometric properties of the configuration space.

\subsection{Equations of Motion}

The kinematics of the dumbbell and asteroid are described in the inertial frame by
\begin{itemize}
    \item \( \vecbf{x} \) - the position of the center of mass of the dumbbell spacecraft represented in the inertial frame \( \vecbf{e}_i\)
    \item \( R \) - the rotation matrix which transforms vectors defined in the spacecraft fixed frame, \( \vecbf{b}_i \), to the inertial frame, \( \vecbf{e}_i \)
    \item \( \vecbf{\Omega} \) - the angular velocity of the spacecraft body fixed frame relative to the inertial frame and represented in the dumbbell body fixed frame \( \vecbf{b}_i \)
    \item \( R_A \) - the rotation matrix which transforms vectors defined in the asteroid fixed frame, \( \vecbf{f}_i \), to the inertial frame, \( \vecbf{e}_i \)
\end{itemize}
In this work, we assume that the asteroid is much more massive than the spacecraft and it's motion is not affected by that of the spacecraft.
This assumption allows us to treat the motion of the vehicle independently from that of the asteroid. 

Using our kinematic variables we can define the kinetic and potential energy as
\begin{align}\label{eq:kinetic_energy}
    T &= \frac{1}{2} m \norm{\dot{\vb{x}}}^2 + \frac{1}{2} \tr{S(\vb{\Omega}) J_d S(\vb{\Omega})^T} , \\
    V( \vecbf{x}, R ) &=  - m_1 U \parenth{R_A^T \parenth{\vecbf{x} + R \vecbf{\rho}_1}} - m_2 U \parenth{R_A^T \parenth{\vecbf{x} + R \vecbf{\rho}_2}} ,
\end{align}
where the polyhedron potential is defined as 
\begin{align}
    U(\vecbf{r}) &= \frac{1}{2} G \sigma \sum_{e \in \text{edges}} \vecbf{r}_e \cdot \vecbf{E}_e \cdot \vecbf{r}_e \cdot L_e - \frac{1}{2}G \sigma \sum_{f \in \text{faces}} \vecbf{r}_f \cdot \vecbf{F}_f \cdot \vecbf{r}_f \cdot \omega_f,
\end{align}
and \( \vecbf{r}_e\) and \(\vecbf{r}_f \) are the vectors from the spacecraft to any point on the respective edge or face, \( G\) is the universal gravitational constant, and \( \sigma \) is the constant density of the asteroid.
The position of each mass \(m_i\) of the dumbbell is defined in the dumbbell fixed frame by the vector \(\vb{\rho}_i\). 
The next step is to define the variations of the kinetic and potential energy to derive the equations of motion.
The variation of the potential and kinetic energy can then be defined as
\begin{align} 
    \delta V &= -\sum_{i=1}^2  m_i \parenth{R_A \deriv{U}{\vb{z}_i} }^T \delta \vb{x} + m_i \hat{\vb{\eta}}\cdot \hat{\vb{\rho}_1} R^T R_A \deriv{U}{\vb{z}_i}, \\
    \delta T &= \parenth{m_1 + m_2} \dot{\vecbf{x}}^T \delta \dot{\vb{x}} + \frac{1}{2} \tr{ - \dot{\hat{\vb{\eta}}} S(J \vb{\Omega}) + \hat{\vb{\eta}} S(\hat{ \vb{\Omega}} J \vb{\Omega})} , 
\end{align}
Finally, applying Hamilton's principle gives the inertial equations of motion of the dumbbell spacecraft as
\begin{align}
    \dot{\vb{x}} &= \vb{v}, \\
    \parenth{m_1 + m_2} \dot{\vecbf{v}} &= m_1 R_A \deriv{U}{\vecbf{z}_1} + m_2 R_A \deriv{U}{\vecbf{z}_2}, \\
    \dot{R} &= R S(\vb{\Omega}) , \\
    J \dot{\vecbf{\Omega}} + \vecbf{\Omega} \times J \vecbf{\Omega} &= \vecbf{M}_1 + \vecbf{M}_2.
\end{align}
The vectors \( \vecbf{z}_1 \) and \( \vecbf{z}_2\) define the position of the dumbbell masses as represented in the asteroid fixed frame and are defined as
\begin{align}
    \vecbf{z}_1 &= R_A^T \parenth{\vecbf{x} + R \vecbf{\rho}_1} , \\
    \vecbf{z}_2 &= R_A^T \parenth{\vecbf{x} + R \vecbf{\rho}_2}.
\end{align}
The gravitational moment on the dumbbell \( \vecbf{M}_i\) is defined as
\begin{align}
    \vecbf{M}_i = m_i \parenth{S(R_A^T \vb{\rho}_i) R^T \deriv{U}{\vb{z}_i}}.
\end{align}
 
In Reference~\citenum{kulumani2016d}, a methodology was developed to design general transfer trajectories about the asteroid 4769 Castalia.
That previous work utilized the polyhedron potential method and a reachability analysis to allow for a spacecraft to transition  between orbits of the asteroid.
\Cref{fig:trajectory} shows an example transfer between two planar equatorial orbits of 4769 Castalia.
Reference~\citenum{kulumani2016d} only considered the orbital dynamics of a point mass spacecraft. 
In this work, we incorporate the attitude dynamics and consider the coupled motion of the spacecraft.
In addition, a desired landing trajectory is followed using nonlinear control techniques.
\subsection{Nonlinear Landing Controller}\label{sec:landing}


\section{Expected Results and Significance}\label{sec:expected}

\bibliographystyle{AAS_publication} 
\bibliography{library}

\end{document}
